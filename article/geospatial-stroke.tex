%pandoc -f latex -t docx  geospatial-stroke.tex --bibliography references.bib -o geospatial.docx --csl=frontiers-in-physics.csl
\documentclass[utf8]{frontiersHLTH}

\usepackage{url,hyperref,lineno,microtype,subcaption}
\usepackage[onehalfspacing]{setspace}
\usepackage{verbatimbox}
\usepackage{mdframed}
\usepackage{color}
\definecolor{dkgreen}{rgb}{0,0.6,0}
\definecolor{gray}{rgb}{0.5,0.5,0.5}
\definecolor{mauve}{rgb}{0.58,0,0.82}
\usepackage{listings}
\lstset{language=R,
basicstyle=\footnotesize,       % the size of the fonts that are used for the code
  numbers=left,                   % where to put the line-numbers
  numberstyle=\tiny\color{gray},  % the style that is used for the line-numbers
  stepnumber=1,                   % the step between two line-numbers. If it's 1, each line
                                  % will be numbered
  numbersep=5pt,                  % how far the line-numbers are from the code
  backgroundcolor=\color{white},  % choose the background color. You must add \usepackage{color}
  showspaces=false,               % show spaces adding particular underscores
  showstringspaces=false,         % underline spaces within strings
  showtabs=false,                 % show tabs within strings adding particular underscores
  frame=single,                   % adds a frame around the code
  rulecolor=\color{black},        % if not set, the frame-color may be changed on line-breaks within not-black text (e.g. commens (green here))
  tabsize=2,                      % sets default tabsize to 2 spaces
  captionpos=b,                   % sets the caption-position to bottom
  breaklines=true,                % sets automatic line breaking
  breakatwhitespace=false,        % sets if automatic breaks should only happen at whitespace
  %keywordstyle=\color{blue},      % keyword style
  keywordstyle=\ttfamily,
  commentstyle=\color{dkgreen},   % comment style
  stringstyle=\color{mauve},      % string literal style
  escapeinside={\%*}{*)},         % if you want to add a comment within your code
  deletekeywords={transform,set,distance,mode,count,as,by,read,csv},
  morekeywords={}            % if you want to add more keywords to the set}
}
\linenumbers

\def\keyFont{\fontsize{8}{11}\helveticabold }
\def\firstAuthorLast{Padgham {et~al.}} %use et al only if is more than 1 author
\def\Authors{Mark Padgham\,$^{1}$, Geoff Boeing\,$^{2}$, David Cooley\,$^{3}$, Nicholas Tierney\,$^{4}$, Michael Sumner\,$^{5}$, Thanh G Phan\,$^{7,8}$ and Richard Beare\,$^{9,10,*}$}
% Affiliations should be keyed to the author's name with superscript numbers and be listed as follows: Laboratory, Institute, Department, Organization, City, State abbreviation (USA, Canada, Australia), and Country (without detailed address information such as city zip codes or street names).
% If one of the authors has a change of address, list the new address below the correspondence details using a superscript symbol and use the same symbol to indicate the author in the author list.
\def\Address{$^{1}$Active Transport Futures, Muenster, Germany\\
$^{2}$School of Public Policy and Urban Affairs, Northeastern University, Boston, Massachusetts, USA\\
$^{3}$Symbolix Pty Ltd, Melbourne, Victoria, Australia \\
$^{4}$Department of Econometrics and Business Statistics, Monash University, Melbourne, Victoria, Australia\\
$^{5}$Australian Antarctic Division, Department of the Environment and Energy, Kingston, Tasmania, Australia\\
$^{7}$Clinical Trials Imaging and Informatics Division of Stroke and Aging Research Group, Monash University, Melbourne, Victoria, Australia\\
$^{8}$Stroke Unit, Monash Medical Centre, Melbourne, Victoria, Australia\\
$^{9}$Department of Medicine, Monash University, Melbourne, Victoria, Australia\\
$^{10}$Developmental Imaging, Murdoch Children's Research Institute, Melbourne, Victoria, Australia}

% The Corresponding Author should be marked with an asterisk
% Provide the exact contact address (this time including street name and city zip code) and email of the corresponding author
\def\corrAuthor{Richard Beare\\Peninsula Clinical School\\Frankston Hospital\\2 Hastings Rd\\Frankston\\Victoria 3199\\Australia}

\def\corrEmail{Richard.Beare@monash.edu}

\begin{document}
\onecolumn
\firstpage{1}

\title[Software tools for geospatial analysis]{A review of software tools, data and services for geospatial analysis of stroke services}

\author[\firstAuthorLast ]{\Authors} %This field will be automatically populated
\address{} %This field will be automatically populated
\correspondance{} %This field will be automatically populated

\extraAuth{}% If there are more than 1 corresponding author, comment this line and uncomment the next one.
%\extraAuth{corresponding Author2 \\ Laboratory X2, Institute X2, Department X2, Organization X2, Street X2, City X2 , State XX2 (only USA, Canada and Australia), Zip Code2, X2 Country X2, email2@uni2.edu}

\maketitle

\section{Introduction}\label{introduction}

Endovascular clot retrieval enables reperfusion following ischemic
stroke and results in dramatic reversal of neurological deficit
(ECR)\cite{berkhemer2015randomized,goyal2016endovascular,goyal2015randomized,campbell2015endovascular,saver2015stent}. The
time window for performing this treatment has changed recently with
the publication of the DEFUSE3 and DAWN trials. These trials have
pushed the boundary for performing ECR to 24
hours\cite{nogueira2018thrombectomy}. ECR treatment requires
specialist centers with 24-hour staffing by skilled stroke teams and
interventional radiologists with unrestricted access to angiography
suites. Key decisions for governments and policy makers include: how
many centers are required to service a specified area, where should
the centers be placed and what is the expected load on the
centers\cite{Phan_2017}.Many factors influence the final choice,
including the availability of trained interventional neuroradiologists
(INR) , the number of cases required for INR to retain skills, and
costs.

Related to the above considerations is the issue of transport
model. One proposed model is direct transport to mothership (ECR hub)
where by the ambulance bypass the smaller hospitals including those
which are thrombolysis capable and take the patient to directly to the
ECR hub \cite{Milne_2017, 10.1001/jamaneurol.2018.2424}. The
alternative model is the drip and ship model whereby patients are
brought to the nearest thrombolysis centre for advanced imaging and
triaging on the need for thrombolytic drug and or ECR. A drawback to
the drip and ship model is that there is an additional 99 minutes
delay related to a second transfer of the case to the mothership for
cases with large vessel occlusion (LVO)
\cite{froehler2017interhospital}. A potential issue with taking
patients direct to the mothership is that not all patients with
ischemic stroke require ECR or event TPA. Identification of patients
for ECR at the pre-hospital level requires either the use of LVO scale
or mobile stroke unit (MSU). However, the impact of the use of LVO
scale on ECR hub loading (handling of large volume of ECR and non-ECR
cases) has yet to be tested.

In the writing of this introduction, we have asked several data
scientists (software engineers) to contribute to vignettes to help
clinicians and other citizen scientists aspiring to work in the
field. Both codes and data are provided so that once {\em R} and {\em
  Python} softwares are installed the scientist can copy and paste the
example codes to test run. These softwares are extremely flexible, but
typically involve relatively steep learning curves. The examples were
written with instructions to ease the introduction into geospatial
analysis. The examples provided here are not exhaustive and are
intended to stimulate creative use of these geospatial tools. The two
examples we present are a choropleth (thematic map) and a service
catchment basin estimation. A choropleth is a thematic map display in
which regions are coloured by a measure of the region. We use
demographic and boundary data from the Australian Bureau of Statistics
and incidence data from the NEMISIS
\cite{thrift_stroke_2000,azarpazhooh2008patterns} study to estimate
stroke cases per postcode and display the result on an interactive
map. The service catchment basin estimation involves a Monte-Carlo
simulation of patients attending a rehabilitation service of 3
hospitals. The catchment basin of each hospital is the region that has
lower travel time to that hospital than any other. Catchment basins
can be combined with incidence data to estimate load on rehabilitation
centers. The data can be used to explore scenarios, such as the
removal or addition of service centers.

%Geospatial approaches have been used to %analyse the delivery of
%emergency clot retrieval services %\cite{Phan_2017} and to evaluate
%``Drip and Ship'' approaches in %specific locales \cite{Milne_2017}
%and population level %access to services
%\cite{Adeoye_2014}. %Geospatial tools can be used to analyse and
%visualise geospatial data, %such as patient collection location as
%well as perform a range of %simulations at varying levels of
%detail. For example, in %\cite{Phan_2017}, travel times between a set
%of randomly generated %addresses and a set of possible destinations
%were estimated using %queries to several Google Application
%Programming Interfaces (API), %allowing various configuration of the
%treatment network to be %tested. Combination of the resulting
%catchment areas with demographic %data allowed loadings to be
%estimated. %The studies cited above were constructed using a series
%of standard %geospatial analysis components. In this article we will
%introduce %these components and provide examples of how they can be
%used to %answer health related questions. Examples are implemented
%using open %source software, specifically R and Python, and source
%code provided %so that readers can reproduce and modify them
%%\cite{R_Core_Team_2018,sanner1999python,boeing_osmnx_2017}. Geospatial
%analysis tools %have traditionally been the domain of specialist
%commercial software %and vendors, however this is no longer the case,
%with a range of open %source options available to researchers. These
%tools are extremely %flexible, but typically involve relatively steep
%learning curves. We %hope that his article will provide stroke
%researchers with a useful %introduction to the possibilities offered
%by these tools.

\section{Background-Geospatial Analysis}\label{background} 

In this article we will introduce these components and provide
examples. The article seeks to introduce geospatial methods and to
enable the curious clinician and citizen scientist to become involved
in exploring models of stroke care in their regions. Alternatively, a
deeper understanding of these geospatial methods will enable
clinicians to collaborate with other researchers or citizen scientists
on models to improve local stroke service. We illustrate this in terms
of code for performing the tasks (modelling catchment of
rehabilitation hospitals following treatment in hospitals providing
acute stroke care) and location of geospatial data and boundary
(shapefiles) data in different part of the world. In the absence of
geospatial data on observed cases with stroke in each suburb, we
provide a method for performing the task using data from
epidemiological studies \cite{Phan_2017}. The codes used here are for
two free and open source software packages - {\em R} and {\em
  Python}. Geospatial analysis can also be performed with
Matlab\cite{Milne_2017}; this software is not free and will not be
discussed further.

{\em Geospatial analysis} or modelling of spatial data has
traditionally been the domain of {\em geographic information systems
  (GIS)} specialists, employing commercial software and data
products. Recent years, however, have seen the development of open
source tools and free or low cost web services, such as Google Maps,
that make advanced geospatial analysis accessible and feasible to the
non-specialist citizen scientist. In this article we introduce a
family of computational techniques and services, collectively termed
geospatial analysis tools, that can be applied to a range of questions
relevant to stroke services. Geospatial analysis tools allow
manipulation and modelling of geospatial data. These tools, data, and
modelling techniques have a long track record in the quantitative
geography, city and regional planning, and civil engineering research
literatures. Geospatial data, in the context of stroke research,
includes the location of patients and treatment centers, routes
through the road network linking patients to treatment centers,
geographic and administrative region boundaries (e.g.~post codes,
government areas, national boundaries) and disease incidence and
demographic information associated with such regions.

\subsection{Geospatial frameworks}\label{geospatial-frameworks} 
In geospatial analysis, location of a data point in space is referred
to in terms of longitude and latitude. In practice, longitude is the X
axis and latitude is the Y axis. More complex data, such as national
boundaries or administrative or postcode boundaries consist of sets of
points connected together in defined orders, typically to produce a
closed shape. Other structures, such as road networks, are also
constructed using sets of points and include other types of
information, such as speed limits, travel direction etc. A geospatial
framework provides mechanisms for representing, loading, and saving
geospatial data and performing fundamental mathematical
operations. For example, the simple features (sf) \cite{Pebesma_2018}
package, on which our R examples are based, provides structures to
represent all manner of shapes and associate them with non-spatial
quantities, perform transforms between coordinate systems, display
shapes, compute geometric quantities like areas and distances and
perform operations like intersections and unions. The equivalent
Python framework is the geopandas package that provides a geospatial
extension to standard data frames. A key emerging subdomain of
geospatial analysis is spatial network analysis. Several open-source
packages now exist for modelling and analyzing spatial networks, such
as urban street networks, including dodgr for R \cite{Padgham_2019}
and OSMnx for Python \cite{boeing_osmnx_2017}.

\subsection{Sources of regional data}\label{sources-of-regional-data} 

The examples below use postcode boundary data available from the
Australian Bureau of Statistics (the codes for performing this task
are available in the supplementary material or on the website
\href{https://richardbeare.github.io/GeospatialStroke/}{https://richardbeare.github.io/GeospatialStroke/}). It
is common for boundaries used in reporting of regional statistics to
be available in standard file formats from the reporting bodies or
central authorities along with the reported statistics. The regional
demographics measures, often derived from national census data, also
represent an important source of information for researchers,
including age, sex, income, ethnicity etc. For example, in the US, key
data sources on sociodemographics and the built environment include
the Census Bureau's decennial census \cite{us_census_bureau_decennial}
(a complete enumeration at fine spatial scales but coarse, decadal
temporal scales), American Community Survey\cite{us_census_bureau_acs}
(a survey with annual temporal scales, but often fairly large standard
errors at small spatial scales due to the sample size), and TIGER/Line
shapefiles\cite{us_census_tiger_line} of tract, municipal, and
urbanized area boundaries. A comprehensive repository of US road
network models at regional and municipal scales is available on the
Harvard Dataverse \cite{boeing_street_2019}. Additional regional data
are frequently available from municipal, state, county, or
metropolitan governmental agencies. Demographic data for countries in
the European Union are provided by Eurostat \cite{eurostat}. This
includes time series data from several years to decades on economics,
demography, infrastructure, health, traffic, and more of the EU
\cite{Lahti2017}. Geographic data for the EU is available through the
Geographic Information System of the COmmission (GISCO), part of
Eurostat. Similar levels of demographic data are available from France
through INSEE \cite{insee}, Germany through Destatis \cite{destatis}
and, Switzerland through \cite{swiss-bfs}. There are a number of
European sources of geospatial data -
\cite{diva-gis,germany-gis,swiss-3d}.

\subsection{Geocoding and reverse geocoding}\label{geocoding-and-reverse-geocoding} 

Location information, such as a patient home address, is often
available as a street address, rather than a coordinate (a
longitude/latitude pair). However operations, such as plotting
addresses on a map, require a coordinate. Geocoding is the process of
converting an address to a coordinate. Reverse geocoding converts a
coordinate to an address. A coordinate is useful in many other types
of computation, as we shall see in the examples below. There are two
common approaches to geocoding and reverse geocoding. The most
ubiquitous is via web services such as Google Maps. Other services,
such as OpenStreetMap's Nominatim web service and OpenCage
(\url{https://opencagedata.com/}), provide similar capabilities and
all can be queried in a automated way from R and Python
\cite{opencage}. The other approach is via a local database of
geocoded addresses. One example, for Australia, is the PSMA (formerly
Public Sector Mapping Agencies) address database available in an R
queryable form. A local database allows many high speed queries, but
is often less flexible in terms of query structure than the web
services. Web services are discussed in more detail below.

\subsection{Distance and travel estimation}\label{distance-and-travel-estimation} 

A key part of a number of studies cited above is the estimation of
travel time between patient and treatment center. The popularity of
personal navigation systems in smartphones has driven the development
of extremely sophisticated tools to estimate the fastest route between
points. One of the best known, Google Maps \footnote{the two APIs
  involved are the directions api and distance api}, uses a
combination of information about the road network, historic travel
time data derived from smartphone users and live information from
smartphones. The travel time estimates are thus sensitive to time of
day, weather conditions and possibly traffic accidents. Google, and
other web services for travel time estimation, can be queried in a
similar fashion to the geocoding services. It is also possible to
create a local database to represent the road network, allowing more
rapid querying, but losing some of the benefits of traffic models.

\subsection{Visualization}\label{visualization} 

Two forms of visualization are used in the following examples - static and interactive. Static maps are required for printed reports and typically present a carefully selected view. Interactive maps allow exploration of a data set, via zooming and toggling of overlays. Interactive maps often use web services to provide the background map ``tiles'', over which data is superimposed. Different interactive web services specialise in different types of display. Some tools produce static and interactive displays in very similar ways. 

\subsection{Introduction to web services}\label{introduction-to-web-services}}
 Web services
Web services providing various forms of geospatial capabilities are a crucial component of the geospatial analysis tools now available to researchers. Web services deliver what used to be complex and specialised information products to the general public. Geocoding and travel time estimation are two common examples that have already been discussed. Other capabilities include delivery of tiled maps (such as the Google Maps display), street network and building footprint data (such as from OpenStreetMap), and census data on sociodemographic or built environment characteristics (such as from the US Census Bureau's web site). 

\subsubsection{Application programming interfaces (API)}\label{application-programming-interfaces-api} Web services such as Google Maps are accessible via an API. The API allows software tools, such as {\em R} or {\em Python}, to make requests to the web service and retrieve results. Thus, if we consider the Google Maps example, not only can a user access a map query for an address via a web browser, but a program can submit the same request. Furthermore, a program can submit a series of automated requests. For example, given a list of addresses, it is relatively simple to generate an R or Python procedure to geocode all of them via a web service. Many APIs such as Google Maps are commercial products and thus charge for use, although the use is often free for small volumes. The combination of these factors tends to mean that many APIs require somewhat complex setup, typically via signup and creation of keys. Terms of use may evolve over time, with charging being introduced, possibly leading to a need to enter credit card details. We have endeavoured to create examples that do not require keys, simplifying getting started. However, some extensions have been included that do require keys. These are described in supplementary material. 

\subsubsection{OpenStreetMap (OSM)}\label{openstreetmap-osm} OSM (\url{https://www.openstreetmap.org/}) is a service collecting and distributing crowdsourced geospatial data. Many useful OSM services are available without API keys, and it is thus the platform of choice for examples in this paper. OSM is also unusual in that it allows access to geospatial structures, such as road networks, rather than images generated from those structures. This capability is used to estimate travel time. 

\subsubsection{Access to the examples}\label{access-to-the-examples} The examples are available in their source code form from \href{https://github.com/richardbeare/GeospatialStroke}{https://github.com/richardbeare/GeospatialStroke}. ``Live'' versions and instructions are available at \href{https://richardbeare.github.io/Geospatial/index.html}{https://richardbeare.github.io/Geospatial/index.html} and can be viewed in conjunction with the methods section. The description focuses on the {\em R} versions of the examples. Code is visible in the shaded boxes, while output of the code, such as maps, are displayed immediately after the code. {\em Python} versions are provided and implement equivalent steps. Details on downloading and running the examples are available in supplementary material and at the web site. 

\section{Methods}\label{methods} 
\subsection{Example 1: Choropleth to visualize estimated stroke numbers}\label{example-1-choropleth-to-visualize-estimated-stroke-numbers} 

\subsubsection{Overview:}\label{overview} We demonstrate accessing and using different data sources. The first is Australian Bureau of Statistics census data provided at the postcode level for population information, stratified by age, as well as postcode boundary information. The second data source is incidence data from the North East Melbourne Stroke Incidence Study (NEMESIS)\cite{thrift_stroke_2000}. This is combined with the first dataset to estimate per-postcode stroke incidence. We demonstrate geocoding by finding the location of a hospital delivering acute stroke services, and then display postcodes within 15km, colouring each postcode by estimated stroke incidence. The steps involved are described in Tables \ref{tab:exampleA1} and \ref{tab:exampleA2}. 

\subsubsection{Example 2: Service regions for stroke rehabilitation}\label{example-2-service-regions-for-stroke-rehabilitation} In the second example we demonstrate the idea of estimating catchment basins for a set of three service centers (network comprising three rehabilitation hospitals servicing a hospital with an acute stroke service). The idea can be easily extended to more service centers. A catchment basin, or catchment area for a service center is the region that is closer to that service center than any other. The definition of ``closer'' is critical in this calculation, with travel time through the road network being a useful measure for many practical purposes. The approach used in this example involves the sampling of random addresses within a region of interest around the service centers, estimation of travel time from each address to each service center, assignment of addresses to the closest service center, combination of addresses based on service center to form catchment areas. The catchment areas can then be used to estimate loadings on service centers. The steps involved are described in Table \ref{tab:exampleB}. 

\section{Results} Complete versions of the examples are illustrated online at \href{https://richardbeare.github.io/Geospatial/index.html}{https://richardbeare.github.io/Geospatial/index.html} and can be downloaded, run and explored by the reader. 

\subsection{Example 1: Choropleth to visualize estimated stroke numbers} Spatial data, as displayed by an {\em R} session, is illustrated in Supplementary Tables S1 and S2. Supplementary Table S1 shows the hospital coordinates determined via geocoding, while a subset of the combined spatial, demographic and estimated stroke count data is illustrated in Supplementary Table S2. A visualization of straight-line distance between each postcode and the hospital, useful for verifying the calculation is sensible, is illustrated in Figure \ref{fig:DistanceToMMC}. Finally, Figure \ref{fig:choropleth} provides a screenshot of the interactive choropleth featuring postcodes in the vicinity of the hospital coloured by estimated stroke case load. The display can be exported as a web page and viewed interactively, with the ability to pan and zoom and switch display layers on and off. 

\subsection{Example 2: Service regions for stroke rehabilitation} Results of geocoding rehab center addresses is shown in Supplementary Table S3 with the corresponding visualization in Figure \ref{fig:RehabCenterLocations}. A small selection of random addresses is available in Supplementary Table S4 and the complete set is visualized in Figure \ref{fig:RehabCenterRandomAddresses}. Boundaries of postcodes of interest are shown in Figure \ref{fig:RehabCenterPostcodes}. Figures \ref{fig:RehabCenterAddressCatchments}, \ref{fig:RehabCenterPolyCatchments} and \ref{fig:RehabCenterRoadCatchment} show addressed-based, polygon-based and road-based views of the computed catchment basin. Figures \ref{fig:RehabCenterAddressDistanceHex} and \ref{fig:RehabCenterAddressDistance} show alternative visualizations of travel time using a hexagonal height map and a colour coded road network. Finally, allocation of random addresses to rehabilitation centers and estimated case loads per catchment center are available in Tables \ref{tab:rehabrandomassignment} and \ref{tab:rehabcaselaod} 

\section{Discussion}\label{discussion} There are many potential advantages to including geospatial variables such as location of patients and travel times to treatment centers when analysing performance of, or modelling stroke treatment pathways. The ability to accurately model parts of the treatment pathway is important when fairly allocating limited resources, optimising the placement of those resources, forecasting changes in loading in response to change in population distribution and investigating how to best utilise new technology or treatment options. In this article we have introduced some of the fundamental geospatial analysis components and provided reproducible examples using those components in a form that we hope will reduce the steep learning curve for researchers new to the area. In the writing of this paper, we had heeded the call from this special issue of Frontiers in Neurology to provide source codes and enable both stroke researchers and "...citizen scientist to explore ideas in transportation and access to stroke therapy". Recent years have seen increasing use of geospatial analysis techniques, similar to those explored in this paper, in stroke research as well as other health related areas. Two scenarios, with different spatial scales, for provision of acute stroke services are explored in \cite{10.3389/fneur.2019.00150} via an optimization framework, where the costs being optimized are derived from geospatial measures, including travel time. Implications of new technology (MSU) for rural patients is discussed in \cite{10.3389/fneur.2019.00159} and urban application of the same technology is quantitatively analysed in \cite{10.3389/fneur.2019.00331}, with analysis exploiting travel time derived from web services to estimate deployment boundaries. Rather than providing further examples in acute stroke services of which are published in his special topic \cite{10.3389/fneur.2019.00150, 10.3389/fneur.2019.00159, 10.3389/fneur.2019.00331} we had provided examples here in terms of rehabilitation service provided by 3 hospitals attached to an ECR hub (Figures \ref{fig:RehabCenterLocations}-\ref{fig:RehabCenterAddressDistance}). Fair distribution of load across the hospital network is possible while allowing the patients and family unit in close contact, in keeping with the idea of patient centred design. These studies discussed here were conducted to inform decisions under specific assumptions and in relation to individual urban environments (i.e. specific cities). However, they share common analytical ideas and data, namely geospatial analysis tools of the form explored in this paper. Use of these frameworks allow relevant research to be extended to different urban environments and extended as new assessment tools or treatment or transport options become available. For example, the tools described in this paper allow a researcher and citizen scientist to explore how the models relate to their own city as the key data, travel time and center location, can be collected easily from web services \cite{10.1001/jamaneurol.2018.2424,Milne_2017} Furthermore, the analysis could be modified to investigate the effect of a new LVO assessment tool on decision making \cite{10.3389/fneur.2019.00130}. The focus of discussion thus far has been on acute stroke services, however there are applications in many other health areas. For example, effects of traffic conditions on staff recall times for ST elevation myocardial infarction patients has been explored using similar approaches\cite{10.3389/fcvm.2017.00089}. Finally, geospatial analysis can be adapted to evaluation of brain health as seen in a recent study of the associations between brain measures and a neighbourhood walkability index computed from geospatial data \cite{cerin2017associations} 

\subsection{Limitations} In this project, we have provided examples in urban setting only. Modelling of transport in non-urban rural setting can be complex as it may have to take into account air transport.

\section{Conclusion} 
Computational frameworks facilitating analysis of geospatial data are now more accessible than ever before due to the combination of open source software tools, increasing availability of geospatial, demographic and other relevant health data from government and administrative bodies and a plethora of web services offering advanced geospatial data products. These tools are extremely powerful and flexible and offer the potential to address many important questions in stroke treatment. We hope this paper provides a useful introduction to researchers wanting to utilize spatial data. We invite the reader and citizen scientist to take the next step. 
