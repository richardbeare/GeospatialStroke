
\documentclass[utf8]{frontiers_suppmat} % for all articles
\usepackage{url,hyperref,lineno,microtype}
\usepackage[onehalfspacing]{setspace}



% Leave a blank line between paragraphs instead of using \\

\begin{document}
\onecolumn
\firstpage{1}

\title[Supplementary Material]{{\helveticaitalic{Supplementary Material}}}


\maketitle


\section{Supplementary Material}

Supplementary material relating to this article is provided in the form
of R and python source code that can be used to re-create the examples
discussed in the text, as well as interactive versions of the outputs of
those examples. The material is available in two forms - a zip file in the
Frontiers supplementary material section and a ``live'' version accessible
via the \href{https://github.com/SymbolixAu/Geospatial}{github site}.  

\section{Introduction to examples}

The article discusses two examples: 
\begin{itemize}
\item Construction of a choropleth or thematic map.
\item Estimation of catchment basins and case loadings for rehabilitation centres.
\end{itemize}

The supplementary material includes several implementations of these examples, as follows:

\begin{itemize}
\item {\em R} implementations of both that do not require installation of API keys: \href{https://geospatial.symbolixAU.io/}{Choropleth}, \href{https://geospatial.symbolixAU.io/}{Catchment basins}.
\item {\em Python} implementations of both that do not require installation of API keys. \href{https://geospatial.symbolixAU.io/}{Choropleth}, \href{https://geospatial.symbolixAU.io/}{Catchment basins}.
\item An alternative {\em R} implementation of the second example that utilises API keys to access Google services and Mapbox visualization services: \href{https://geospatial.symbolixAU.io/}{Catchment basins with API keys}.
\end{itemize}

The live versions of the examples requiring API keys do not include interactive visualizations. Examples must be
recreated by the user, with their own keys, in order to use the visualization tools.

\subsection{API Keys and tokens}\label{api-keys}

Online services which offer an interface to their applications will
sometimes require use of an API key, or application programming
interface key. This key should be unique for each user, developer or
application making use of the service as it is a way for the provider to
monitor and, where applicable, charge for use.

Two major mapping platforms that require an API key are Google Maps
and Mapbox, both of which are used in the second version of the
catchment basin example. At the time of writing both allow
unrestricted use of the mapping API. However, Google has limits on the
other services it offers such as geocoding and direction services.

Both Google and Mapbox require users create an account.

The required Google API keys may be obtained by following instructions \href{https://developers.google.com/maps/documentation/embed/get-api-key}{provided by Google}.

The required Mapbox token may be obtained by following instructions \href{https://www.mapbox.com/account/access-tokens}{provided by Mapbox}.
\end{document}
